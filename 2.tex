
\section{Задание 2. Формула Тейлора.}

\vspace{10mm}
\textbf{Решение.}

\begin{enumerate}
    \item Формула производной n-ого порядка

    $\displaystyle y = cos^2 \frac{x}{2} = \frac{1 + cos x}{2} = \frac{1}{2} + \frac{1}{2} cos x$

    $y' = \displaystyle -\frac{1}{2} sin x\quad$ 
    $y'' = \displaystyle -\frac{1}{2} cos x\quad$
    $y''' = \displaystyle \frac{1}{2} sin x\quad$
    $y'''' = \displaystyle \frac{1}{2} cos x$

    Видим зацикливание: 
    $y^{4n} = \displaystyle -\frac{1}{2} sin x\quad$ 
    $y^{4n + 1} = \displaystyle -\frac{1}{2} cos x\quad$
    $y^{4n+2} = \displaystyle \frac{1}{2} sin x\quad$
    $y^{4n+3} = \displaystyle \frac{1}{2} cos x$

    \item Формула Тейлора

    $\displaystyle f(x) = f(0) + f'(0)x + \frac{f''(0)x^2}{2!} + \frac{f'''(0)x^3}{3!} + \frac{f''''(0)x^4}{4!} + \dots = 1 - \frac{x^2}{2 * 2!} + \frac{x^4}{2 * 4!} + \dots$

    $\displaystyle f(x) = 1 - \sum_{k=0}^{(n - 2)//4}\frac{x^{4k + 2}}{2 * (4k+2)!} + \sum_{k=0}^{(n - 4)//4}\frac{x^{4k}}{2 * (4k)!} + R_n(x)$

    \item Всякое с $R_n(x)$

    \begin{enumerate}
        \item уравнение касательной
        
        $f'(0) = - \frac{1}{2}sin(0) = 0$. Значит либо функция в окрестности $x_0$ константа, либо она меняет характер монотонности в этой точке. $cos^2 \frac{x}{2}$ не константная функция, значит в окрестости точки 0 она меняет харатер монотонности\\

        \item Пеаон

    $\displaystyle  R_n(x) = o((x-x_o)^n) = x^n$

    $\displaystyle  \triangle y = f(\pm \triangle x) - f(0) = \frac{1}{2} cos \triangle x - \frac{1}{2}$

    $\displaystyle  dy = -\frac{1}{2}sin(0) dx = 0$

    $\displaystyle  \triangle y - dy = \triangle y < - \delta$

    $\displaystyle \frac{1}{2} cos \triangle x - \frac{1}{2} < -0.01$

    $\displaystyle cos \triangle x - 1 < -0.02$

    $\displaystyle cos \triangle x < 0.98$

    $\displaystyle \triangle x < arccos (0.98)$

    Это оценка с одной стороны: где $x > 0$. Аналогично можно сделать слева от нуля, а можно заметить что cos - четная функция и относительно нуля ее график семтричен. Значит радиус окрестности равен 2 * $arccos(0.98)$
    
    \item Лагранж

    $\displaystyle R_1 = -\frac{\frac{1}{2}cos (c)}{2}x^2$

    $\displaystyle x \in (x_0 -dx; x_0 + dx)$

    $\displaystyle x \in (-\frac{\pi}{36}; \frac{\pi}{36})$, $c \in (0; x) \subset ((-\frac{\pi}{36}; \frac{\pi}{36}))$ значит $cos(c) >= 0$ значит $R_1(x) <= 0$ значит $y(x)$ выпукла вверх и лежит под касательной

    \item нужная степень многочлена Тейлора

    $\displaystyle y(x_0 + dx) = \frac{1 + cos (\frac{\pi}{36})}{2}$

    $\displaystyle n = 6: T(x)_6 - y(x_o + dx) = 1 - \frac{{\frac{\pi}{36}}^2}{2 * 2!} + \frac{{\frac{\pi}{36}}^4}{2 * 4!} - \frac{{\frac{\pi}{36}}^6}{2 * 6!} - \frac{1 + cos (\frac{\pi}{36})}{2} < \delta$

    \includegraphics[width=0.55\linewidth]{Screenshot_1.png}

    Таким образом степени 6 хватает чтобы с заданой точностью приблизиться к значению $cos(dx + x_0)$ рядами Тейлора 
    
    \end{enumerate}

    \item График

    \includegraphics[width=1.0\linewidth]{Screenshot_2.png}
    
\end{enumerate}

\clearpage
