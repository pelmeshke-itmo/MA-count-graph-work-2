\section{Задание 2. Формула Тейлора.}

\textbf{Условие.}

Дана функция $\displaystyle y = f(x) = cos^2 \frac{x}{2}, \quad x_0 = 0, \quad dx = 5^\circ, \quad \delta = 0.01, \quad \varepsilon = 10^{-5}$\\
\begin{enumerate}
    \item Найдите формулу её производной $n$–го порядка.
    \item Представьте функцию формулой Тейлора $n$-го порядка в окрестности точки $x_0$. Обозначьте остаточный член $R_n(x)$, если многочлен Тейлора имеет степень $n$.
    \item Используя полученное представление и данные в таблице:
    \begin{enumerate}
        \item Запишите уравнение касательной к функции в точке $x_0$. Сделайте вывод о характере монотонности функции в окрестности точки $x_0$.
        \item Запишите $R_n(x)$ в форме Пеано. Найдите радиус окрестности $x_0$, в каждой
точке которой значение дифференциала $dy$ отличается от приращения функции
$\Delta y$ не более, чем на $\pm \delta$.
        \item Запишите $R_n(x)$ в форме Лагранжа. Определите знак(и) остатка $R_1(x)$ на
интервале $(x_0 - dx, x_0 + dx)$. Сделайте вывод о взаимном расположении
касательной и графика функции. Что можно сказать о выпуклости и перегибах
функции на этом интервале?
        \item Найдите, какой степени $n$ должен быть многочлен Тейлора, чтобы представлять
функцию в точке $x = x_0 + dx$ с точностью $\varepsilon$. Найдите значение функции в
точке $x$ с заданной точностью.
    \end{enumerate}
    \item Постройте графики функции, многочлена Тейлора и остатка. Сверьте с графиками
результаты аналитического исследования. Образец построения здесь.
\end{enumerate}
\vspace{10mm}
\textbf{Решение.}

It is empty but you can fill it!

\textit{Ответ}:  It is empty but you can fill it!
\clearpage